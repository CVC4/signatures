\documentclass{article}

\usepackage{url}
\usepackage{xspace}
\usepackage{xparse}
\usepackage{xcolor}
\usepackage{amsmath, amsfonts, amssymb}
\usepackage{stmaryrd}
\usepackage{mathrsfs}
\usepackage{array}
\usepackage{geometry}
\usepackage{hyperref}
\usepackage{bussproofs}
\EnableBpAbbreviations

% for an infinite page
\geometry{textwidth=500pt, marginparwidth=120pt, footskip=20pt}
% \geometry{textwidth=400pt, marginparwidth=120pt, paperheight=16383pt, textheight=16263pt, footskip=40pt}
% \addtolength{\oddsidemargin}{-.875in}

% Columns with math mode in displaystyle
\newcolumntype{D}[1]{>{\displaystyle} #1 <{}}
\newcolumntype{F}[1]{>{$\displaystyle} #1 <{$}}

\DeclareDocumentCommand{\enum}{ O{n} m O{,\,} O{1} O{}}{#2_{#4}#5#3\dots #3#2_{#1}#5}

\begin{document}

{\sf
  \begin{center}
    \Large --- \textbf{A proof calculus for the BV bitblasting in CVC4} ---\\
  \end{center}
}\vspace{.5in}

\tableofcontents

\section{Proof Calculus}

The proof calculus is extended with rules for representing the bit-blasting of
bit-vector terms. This is independent of other rules for preprocessing, Boolean
and theory reasoning.

\subsection{Bit-blasting}

Judgment for bit-blasting:
$$\mathtt{bbT}\ n\ x\ [\enum[n-1]{x}[;][0]]$$
in which $n$ is a positive number, $x$ is a term of the bit-vector type for size
$n$, and $x_i$, with $x_i\in\{\bot,\top\}$, for $0\leq i< n$, is the Boolean
equivalent of the $i$-th bit of $x$ (from least to most significant). Thus
$[\enum[n-1]{x}[;][0]]$ is the Boolean bit-level interpretation corresponding to the
bit-vector term $x$.\\

\noindent
List of rules for bit-blasting

\subsubsection{Variables and constants}

    \begin{center}
        \AXC{\strut}
        \RL{\textsc{BBVar}}
        \UIC{\strut$\mathtt{bbT}\ n\ x\ [\enum[n-1]{x}[;][0]]$}
        \DP\qquad
        \AXC{\strut}
        \RL{\textsc{BBConst}}
        \UIC{\strut$\mathtt{bbT}\ n\ v\ [\enum[n-1]{v}[;][0]]$}
        \DP
    \end{center}

\subsubsection{Bit-wise operations (\texttt{bvand}, \texttt{bvor},
    \texttt{bvxor},
    \texttt{bvnot})}

    \begin{center}
      \begin{tabular}{c}
        \AXC{\strut$\mathtt{bbT}\ n\ x\ [\enum[n-1]{x}[;][0]]\quad
        \mathtt{bbT}\ n\ y\ [\enum[n-1]{y}[;][0]]$}
        \RL{\textsc{BBAnd}}
        \UIC{\strut$\mathtt{bbT}\ n\ (bvand\ x\ y)\ [x_0 \land y_0; \ldots; x_{n-1} \land y_{n-1}]$}
        \DP\\ \\
        \AXC{\strut$\mathtt{bbT}\ n\ x\ [\enum[n-1]{x}[;][0]]\quad
        \mathtt{bbT}\ n\ y\ [\enum[n-1]{y}[;][0]]$}
        \RL{\textsc{BBOr}}
        \UIC{\strut$\mathtt{bbT}\ n\ (bvor\ x\ y)\ [x_0 \lor y_0; \ldots; x_{n-1} \lor y_{n-1}]$}
        \DP\\ \\
        \AXC{\strut$\mathtt{bbT}\ n\ x\ [\enum[n-1]{x}[;][0]]\quad
        \mathtt{bbT}\ n\ y\ [\enum[n-1]{y}[;][0]]$}
        \RL{\textsc{BBXor}}
        \UIC{\strut$\mathtt{bbT}\ n\ (bvxor\ x\ y)\ [x_0 \oplus y_0; \ldots; x_{n-1} \oplus y_{n-1}]$}
        \DP\\ \\
        \AXC{\strut$\mathtt{bbT}\ n\ x\ [\enum[n-1]{x}[;][0]]\quad
        	\mathtt{bbT}\ n\ y\ [\enum[n-1]{y}[;][0]]$}
        \RL{\textsc{BBXnor}}
        \UIC{\strut$\mathtt{bbT}\ n\ (bvxnor\ x\ y)\ [x_0 \leftrightarrow y_0; \leftrightarrow; x_{n-1} 
        	\leftrightarrow y_{n-1}]$}
        \DP\\ \\
        \AXC{\strut$\mathtt{bbT}\ n\ x\ [\enum[n-1]{x}[;][0]]$}
        \RL{\textsc{BBNot}}
        \UIC{\strut$\mathtt{bbT}\ n\ (bvnot\ x)\ [\neg x_0; \ldots; \neg x_{n-1}]$}
        \DP\\
      \end{tabular}
    \end{center}

\subsubsection{Arithmetic operations (addition, negation, multiplication)}

\paragraph{Addition}

    Bit-blasted following the \emph{ripple carry adder} way of
    computing:

 \[
      \AXC{\strut$\mathtt{bbT}\ n\ x\ [\enum[n-1]{x}[;][0]]\quad \mathtt{bbT}\ n\
        y\ [\enum[n-1]{y}[;][0]]$} \RL{\textsc{BBAdd}} \UIC{\strut$\mathtt{bbT}\ n\
        (bvadd\ x\ y)\ [(x_0 \oplus y_0)\oplus\mathrm{carry}_0; \ldots; (x_{n-1}
        \oplus y_{n-1})\oplus\mathrm{carry}_{n-1}]$} \DP
    \]
    in which, for $i\geq 0$,
    \[
      \begin{array}{lcl}
        \mathrm{carry}_0&=&\bot\\
        \mathrm{carry}_{i+1}&=&(x_i\wedge y_i)\vee((x_i\oplus y_i)\wedge \mathrm{carry}_i)
      \end{array}
    \]

\paragraph{Negation}

      Bit-blasted following $(\mathit{bvneg}\ a)\equiv (\mathit{bvadd}\
      (\mathit{bvnot}\ a)\ 1)$:
      \[
        \AXC{\strut$\mathtt{bbT}\ n\ x\ [\enum[n-1]{x}[;][0]]$} \RL{\textsc{BBNeg}}
        \UIC{\strut$\mathtt{bbT}\ n\ (bvneg\ x)\ [(\neg x_0 \oplus
          \bot)\oplus\mathrm{carry}_0; \ldots; (\neg x_{n-1}
          \oplus\bot)\oplus\mathrm{carry}_{n-1}]$} \DP
      \]
      in which, for $i\geq 0$,
      \[
        \begin{array}{lcl}
          \mathrm{carry}_0&=&\top\\
          \mathrm{carry}_{i+1}&=&(\neg x_i\wedge \bot)\vee((\neg x_i\oplus
                                  \bot)\wedge \mathrm{carry}_i)
        \end{array}
      \]


\paragraph{Multiplication}

      Bit-blasted following the \emph{shift add multiplier} way of computing:

      \[
        \AXC{\strut$\mathtt{bbT}\ n\ x\ [\enum[n-1]{x}[;][0]]\quad \mathtt{bbT}\ n\
          y\ [\enum[n-1]{y}[;][0]]$} \RL{\textsc{BBMult}} \UIC{\strut$\mathtt{bbT}\
          n\ (bvmult\ x\ y)\ [\mathrm{res}^{n-1}_0; \ldots; \mathrm{res}^{n-1}_{n-1}]$} \DP
      \]
      in which, for $i,j\geq 0$,
      \[
        \begin{array}{lcl}
          \mathrm{res}^0_i&=&\mathrm{sh}^0_i\\[1ex]
          \mathrm{res}^j_0&=&\mathrm{sh}^0_0\\[1ex]
          \mathrm{res}^{j+1}_i&=&\mathrm{res}^j_i\oplus\mathrm{sh}^{j+1}_i\oplus\mathrm{carry}^{j+1}_i\\[1ex]
          \mathrm{carry}^0_i&=&\bot\\[1ex]
          \mathrm{carry}^{j+1}_{i+1}&=&\left\{
                                        \begin{tabular}{F{l}l}
                                          \mathrm{res}^j_i\wedge\mathrm{sh}^{j+1}_i\vee((\mathrm{res}^j_i\oplus\mathrm{sh}^{j+1}_i)\wedge\mathrm{carry}^{j+1}_i)&if $j < i$\\
                                          \bot&otherwise\\
                                        \end{tabular}\right.\\[3ex]
          \mathrm{sh}^j_i&=&\left\{
                             \begin{tabular}{F{l}l}
                               x_{i-j}\wedge y_j&if $j \leq i$\\
                               \bot&otherwise\\
                             \end{tabular}\right.
        \end{array}
      \]
      Example mult for $n = 4$:
      \[
        \left[
          \begin{tabular}{F{l}F{c}F{l}}
            \mathrm{res}^3_0&=&(x_0 \wedge y_0);\\
            \mathrm{res}^3_1&=&\underbrace{(x_1\wedge y_0)\oplus(x_0\wedge
                                          y_1)}\oplus\underbrace{\mathrm{carry}^3_1};\\
                            &&\qquad\,\underbrace{\mathrm{res}^0_1\oplus\mathrm{sh}^1_1}\qquad\qquad\
                               \bot\\
                            &&\qquad\quad\ \,\mathrm{res}^2_1\\
            \mathrm{res}^3_2&=&\underbrace{(x_2\wedge y_0)\oplus(x_1\wedge y_1)}\oplus\underbrace{\mathrm{carry}^1_2}\oplus\underbrace{(x_0\wedge
                                          y_2)}\oplus\underbrace{\mathrm{carry}^3_2};\\
                            &&\qquad\,\underbrace{\mathrm{res}^0_2\oplus\mathrm{sh}^1_2}\qquad\mathrm{res}^0_1\wedge\mathrm{sh}^1_1\dots\quad\mathrm{sh}^2_2\qquad\quad\bot\\
                            &&\qquad\qquad\underbrace{\mathrm{res}^1_2\qquad\qquad\qquad\qquad}\\
                            &&\qquad\qquad\qquad\qquad\mathrm{res}^2_2\\
             \mathrm{res}^3_3&=&\underbrace{(x_3\wedge y_0)\oplus(x_2\wedge y_1)}\oplus\underbrace{\mathrm{carry}^1_3}\oplus\underbrace{(x_1\wedge
                                          y_2)}\oplus\underbrace{\mathrm{carry}^2_3}\oplus\underbrace{(x_0\wedge y_3)}\oplus\underbrace{\mathrm{carry}^3_3};\\
                            &&\qquad\
                               \underbrace{\mathrm{res}^0_3\oplus\mathrm{sh}^1_3\qquad\qquad
                               \dots}\qquad\quad\mathrm{sh}^2_3\ \ \ \mathrm{res}^1_2\wedge\mathrm{sh}^2_2\dots\quad\mathrm{sh}^3_3\qquad\quad\bot\\
                            &&\qquad\qquad\qquad\underbrace{\mathrm{res}^1_3\qquad\qquad\qquad\qquad\qquad\qquad\qquad}\\
            &&\qquad\qquad\qquad\qquad\qquad\qquad\quad\mathrm{res}^2_3
          \end{tabular}
        \right]
      \]
      with
      \[
      \begin{tabular}{F{l}cF{l}l}
        \mathrm{carry}^1_3&=&\mathrm{res}^0_2\wedge\mathrm{sh}^1_2\vee((\mathrm{res}^0_2\oplus\mathrm{sh}^1_2)\wedge\mathrm{carry}^1_2)\\[1ex]
\mathrm{carry}^1_2&=&\mathrm{res}^0_1\wedge\mathrm{sh}^1_1\vee((\mathrm{res}^0_1\oplus\mathrm{sh}^1_1)\wedge\underbrace{\mathrm{carry}^1_1})\\
        &&\qquad\qquad\qquad\qquad\qquad\qquad\quad\!\bot\\[1ex]
\mathrm{carry}^2_3&=&\mathrm{res}^1_2\wedge\mathrm{sh}^2_2\vee((\mathrm{res}^1_2\oplus\mathrm{sh}^2_2)\wedge\underbrace{\mathrm{carry}^2_2})\\
                          &&\qquad\qquad\qquad\qquad\qquad\qquad\quad\!\bot\\[1ex]
      \end{tabular}
\]

\subsubsection{Extraction}

      \[
       \AXC{\strut$\mathtt{bbT}\ n\ x\ [\enum[n-1]{x}[;][0]]\quad 0\leq j \leq
         i< n$}
       \RL{\textsc{BBExtract}}
       \UIC{\strut$\mathtt{bbT}\ i-j+1\ (extract\ i\ j\ x)\ [ x_j; \ldots; x_i]$}
       \DP
     \]

\subsubsection{Concatenation}

      \[
       \AXC{\strut$\mathtt{bbT}\ n\ x\ [\enum[n-1]{x}[;][0]]\quad
          \mathtt{bbT}\ m\ y\ [\enum[m-1]{y}[;][0]]$}
       \RL{\textsc{BBConcat}}
       \UIC{\strut$\mathtt{bbT}\ n+m\ (concat\ x\ y)\ [ 
       	\enum[n-1]{x}[;][0]; \enum[m-1]{y}[;][0]]$}
       \DP
     \]

\subsubsection{Equality}
      \[
       \AXC{\strut$\mathtt{bbT}\ n\ x\ [\enum[n-1]{x}[;][0]]\quad
          \mathtt{bbT}\ n\ y\ [\enum[n-1]{y}[;][0]]$}
       \RL{\textsc{BBEq}}
       \UIC{\strut$(bveq\ x\ y)\ \leftrightarrow\ (x_0 \leftrightarrow y_0\ \land\ \ldots\ \land\ x_{n-1} \leftrightarrow y_{n-1})$}
       \DP
      \]

\subsubsection{Comparison predicates (signed/unsigned)}

\paragraph{Unsigned less than}


Bit-blasted following $a < b$ iff, for $0\leq i<n$, $(\sim a[i]\mbox{ AND
}b[i])\mbox{ OR }( a[i] \leftrightarrow
b[i]\mbox{ AND }a[i+1:n] < b[i+1:n])$

\[
  \AXC{\strut$\mathtt{bbT}\ n\ x\ [\enum[n-1]{x}[;][0]]\quad
    \mathtt{bbT}\ n\ y\ [\enum[n-1]{y}[;][0]]$}
  \RL{\textsc{BBUlt}}
  \UIC{\strut$(bvult\ x\ y)\ \leftrightarrow\
    \mathrm{res}_{n-1}$}
  \DP
\]
in which, for $i\geq 0$,
\[
  \begin{array}{lcl}
    \mathrm{res}_0&=&\neg x_0 \wedge y_0\\
    \mathrm{res}_{i+1}&=&((x_{i+1}\leftrightarrow y_{i+1})\wedge
                          \mathrm{res}_i)\vee (\neg x_{i+1}\wedge y_{i+1})
  \end{array}
\]

\paragraph{Signed less than}

Bit-blasted following $a < b$ iff $a$ is negative and $b$ positive, or they
have the same sign and the value of $a$ is less than the value of $b$ (i.e.\ the
same as above)

\[
  \AXC{\strut$\mathtt{bbT}\ n\ x\ [\enum[n-1]{x}[;][0]]\quad
    \mathtt{bbT}\ n\ y\ [\enum[n-1]{y}[;][0]]$}
  \RL{\textsc{BBSlt}}
  \UIC{\strut$(bvslt\ x\ y)\ \leftrightarrow\ (x_{n-1}\wedge\neg y_{n-1})\vee
    \left((x_{n-1}\wedge y_{n-1}) \wedge
    \mathrm{res}_{n-2}\right)$}
  \DP
\]
in which, for $i\geq 0$,
\[
  \begin{array}{lcl}
    \mathrm{res}_0&=& x_0 \wedge \neg y_0\\
    \mathrm{res}_{i+1}&=&((x_{i+1}\leftrightarrow y_{i+1})\wedge
                          \mathrm{res}_i)\vee (x_{i+1}\wedge \neg y_{i+1})
  \end{array}
\]

\subsubsection{Signed extension}

    \begin{center}
        \AXC{\strut$\mathtt{bbT}\ n\ x\ [\enum[n-1]{x}[;][0]]$}
        \RL{\textsc{BBSExt}}
        \UIC{\strut$\mathtt{bbT}\ n+k\ (bbsextend\ k\ x)\ [\enum[n-1]{x}[;][0];\enum[n+k-1]{x}[;][n]]$}
        \DP
    \end{center}
\noindent
in which $\enum[n+k-1]{x}[;][n][=x_{n-1}]$, i.e., all the extension bits are equal
to the signed bit (the most significant one in the original bit-vector).


% \subsection{Arrays}

%   \[
%     \begin{array}{c}
%     \AXC{\strut}
%     \RL{\textsc{ReadOverWrite$_{Same}$}}
%     \UIC{\strut$a[\,i \leftarrow v\,][\,i\,]\ =\ v$}
%     \DP\\[2em]
%     \AXC{\strut}
%     \RL{\textsc{ReadOverWrite$_{Other}$}}
%     \UIC{\strut$a[\,i \leftarrow v\,][\,j\,]\ =\ a[\,j\,]$}
%     \DP\\[2em]
%     \AXC{\strut$a \neq b$}
%     \RL{\textsc{Extensionality}}
%     \UIC{\strut$\exists k.\ a[\,k\,]\ \neq\ b[\,k\,]$}
%     \DP
%     \end{array}
%   \]

\section{veriT proof format}

Proofs involving bit-vector reasoning will have an overall structure with four
components:
\begin{enumerate}
  \item Inputs
  \item Definition of bit-vector terms (via the above bit-blasting rules)
  \item Resolution subproof of each bit-vector lemma learned

  The subproof will depend on the bit-blasting definitions and Boolean reasoning,
  such that one can derive e.g.\ the following valid clause
\begin{verbatim}
(resolution ((not (= #b1 a)) (not (= #b0 b)) (not (= #b1 c)) (= (bvadd a b) c)) n1...nk)
\end{verbatim}
  in which \texttt{n1}, ..., \texttt{nk} will be the Boolean reasoning clauses.

  \item Resolution proof using the learned bit-vector lemmas and inputs to
  derive a conflict.
\end{enumerate}


\subsection{Bit-blasting}

Examples of \textsc{BBVar} and \textsc{BBConst}:

\begin{verbatim}
(bbvar ((bbT a [ (bitof 0 a) (bitof 1 a) (bitof 2 a) (bitof 3 a)])))
(bbconst ((bbT #b0010 [ false true false false])))
\end{verbatim}

\noindent
Example of \textsc{BBEq}:

\begin{verbatim}
(bbeq ((=
         (= #b0010 a)
         (and
           (= false (bitof 0 a))
           (and
             (= true (bitof 1 a))
             (and
               (= false (bitof 2 a))
               (= false (bitof 3 a))))))) n1 n2)
\end{verbatim}
in which \texttt{n1} is the definition of \texttt{\#b0010} via \textsc{BBConst}
and \texttt{n2} the definition of \texttt{a} via \textsc{BBvVar}.

% \begin{verbatim}
% (bbvar ((bbT (select a (select b (select c v1)))
%         [ (bitof 0 (select a (select b (select c v1))))
%           (bitof 1 (select a (select b (select c v1))))
%           (bitof 2 (select a (select b (select c v1))))
%           (bitof 3 (select a (select b (select c v1))))])))

% (bbvar ((bbT v0 [ (bitof 0 v0) (bitof 1 v0) (bitof 2 v0) (bitof 3 v0)])))
% \end{verbatim}
\end{document}

%%% Local Variables:
%%% mode: latex
%%% TeX-master: t
%%% End:
